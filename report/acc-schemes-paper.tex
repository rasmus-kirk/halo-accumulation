% Options for packages loaded elsewhere
\PassOptionsToPackage{unicode}{hyperref}
\PassOptionsToPackage{hyphens}{url}
\PassOptionsToPackage{dvipsnames,svgnames,x11names}{xcolor}
%
\documentclass[
]{article}
\usepackage{amsmath,amssymb}
\usepackage{iftex}
\ifPDFTeX
  \usepackage[T1]{fontenc}
  \usepackage[utf8]{inputenc}
  \usepackage{textcomp} % provide euro and other symbols
\else % if luatex or xetex
  \usepackage{unicode-math} % this also loads fontspec
  \defaultfontfeatures{Scale=MatchLowercase}
  \defaultfontfeatures[\rmfamily]{Ligatures=TeX,Scale=1}
\fi
\usepackage{lmodern}
\ifPDFTeX\else
  % xetex/luatex font selection
\fi
% Use upquote if available, for straight quotes in verbatim environments
\IfFileExists{upquote.sty}{\usepackage{upquote}}{}
\IfFileExists{microtype.sty}{% use microtype if available
  \usepackage[]{microtype}
  \UseMicrotypeSet[protrusion]{basicmath} % disable protrusion for tt fonts
}{}
\makeatletter
\@ifundefined{KOMAClassName}{% if non-KOMA class
  \IfFileExists{parskip.sty}{%
    \usepackage{parskip}
  }{% else
    \setlength{\parindent}{0pt}
    \setlength{\parskip}{6pt plus 2pt minus 1pt}}
}{% if KOMA class
  \KOMAoptions{parskip=half}}
\makeatother
\usepackage{xcolor}
\usepackage[margin=2cm]{geometry}
\setlength{\emergencystretch}{3em} % prevent overfull lines
\providecommand{\tightlist}{%
  \setlength{\itemsep}{0pt}\setlength{\parskip}{0pt}}
\setcounter{secnumdepth}{-\maxdimen} % remove section numbering
\usepackage{xcolor}
\definecolor{GbGreenNt}{HTML}{98971a}
\definecolor{GbBlueNt}{HTML}{458588}
\definecolor{GbGrayNt}{HTML}{928374}
\definecolor{GbBlueDk}{HTML}{076678}
\definecolor{GbGrey}{HTML}{504945}
\definecolor{GbFg4}{HTML}{7c6f64}
\definecolor{GbFg3}{HTML}{665c54}
\definecolor{GbFg2}{HTML}{504945}
\definecolor{GbFg1}{HTML}{3c3836}
\usepackage{mathtools}

\usepackage{subcaption} 
\usepackage{tikz}
\usetikzlibrary{shapes.geometric, arrows, positioning}

\tikzstyle{process} = [rectangle, minimum width=2cm, minimum height=1cm, text centered, draw=black]
\tikzstyle{data} = [trapezium, trapezium left angle=70, trapezium right angle=110, minimum width=1.5cm, minimum height=1cm, text centered, draw=black]
\tikzstyle{arrow} = [->, >=latex]
\tikzstyle{thick-arrow} = [thick, ->, >=latex]
\tikzstyle{dashed-arrow} = [dashed, ->]
\tikzstyle{node} = [rectangle, draw=black, minimum width=1cm, minimum height=1cm, text centered]

\tikzstyle{box} = [rectangle, draw, rounded corners, text centered, minimum height=1cm]

\usepackage{pgfplots}
\pgfplotsset{compat=1.18}
\usepackage{pgfplotstable}

\usepackage{hyperref}
\hypersetup{
    colorlinks=true, %set true if you want colored links
    linktoc=all,     %set to all if you want both sections and subsections linked
    linkcolor=GbBlueNt,    % Internal links (e.g., section links)
    citecolor=GbGreenNt,   % Citation links
    urlcolor=GbBlueDk      % External URL links
}

\usepackage{caption}
\usepackage{algorithm}
\usepackage{algpseudocode}

\newcommand{\Desc}[2]{\hspace*{\algorithmicindent} \makebox[14.5em][l]{#1}\parbox[t]{32em}{#2}}

\usepackage[style=iso]{datetime2}

% Float figures
\usepackage{float}
\let\origfigure\figure
\let\endorigfigure\endfigure
\renewenvironment{figure}[1][2] {
    \expandafter\origfigure\expandafter[H]
} {
    \endorigfigure
}

\newcommand*\Bb{\mathbb{B}}
\newcommand*\Zb{\mathbb{Z}}
\newcommand*\Fb{\mathbb{F}}
\newcommand*\Nb{\mathbb{N}}
\newcommand*\Rb{\mathbb{R}}
\newcommand*\Eb{\mathbb{E}}
\newcommand*\Gb{\mathbb{G}}
\newcommand*\Ac{\mathcal{A}}
\newcommand*\Bc{\mathcal{B}}
\newcommand*\Cc{\mathcal{C}}
\newcommand*\Dc{\mathcal{D}}
\newcommand*\Ec{\mathcal{E}}
\newcommand*\Rc{\mathcal{R}}
\newcommand*\Oc{\mathcal{O}}
\newcommand*\Uc{\mathcal{U}}
\newcommand*\Mc{\mathcal{M}}
\newcommand*\Pc{\mathcal{P}}
\newcommand*\Vc{\mathcal{V}}
\newcommand*\Sc{\mathcal{S}}
\newcommand*\Hc{\mathcal{H}}

\renewcommand*\a{\alpha}
\renewcommand*\b{\beta}
\renewcommand*\d{\delta}
\newcommand*\e{\epsilon}
\renewcommand*\l{\lambda}
\newcommand*\p{\phi}
\renewcommand*\o{\omega}
\newcommand*\ps{\psi}
\renewcommand*\S{\Sigma}

\renewcommand*\mod{\bmod}
\newcommand*\cat{\mathbin{+\mkern-10mu+}}
\newcommand*\bor{\mathbin{\&\mkern-7mu\&}}
\newcommand*\xor{\oplus}
\newcommand*\meq{\stackrel{?}{=}}
\newcommand*\iso{\cong}
\newcommand{\qed}{\hfill \ensuremath{\Box}}
\newcommand{\defend}{\hfill \ensuremath{\triangle}}
\newcommand*{\then}{\implies}

\newcommand*\algind{\hspace*{\algorithmicindent}}
\newcommand*\algindd{\algind \algind}

\newcommand{\textblue}[1]{\textcolor{GbBlueDk}{#1}}
\newcommand{\mathblue}[1]{\mathcolor{GbBlueDk}{#1}}
\newcommand{\textgrey}[1]{\textcolor{GbGrey}{#1}}
\newcommand{\mathgrey}[1]{\mathcolor{GbGrey}{#1}}
\newcommand{\floor}[1]{\left \lfloor #1 \right \rfloor }
\newcommand{\ceil}[1]{\left \lceil #1 \right \rceil }
\renewcommand{\vec}[1]{ \boldsymbol{#1} }
\newcommand{\ran}[1]{ \mathrm{#1} }
\newcommand{\ranvec}[1]{ \boldsymbol{\ran{#1}} }
\newcommand{\dotp}[2]{ \langle #1, #2 \rangle }
\newcommand{\ip}[2]{ \langle #1, #2 \rangle }
\newcommand{\IK}[2]{ \mathrm{I.K} }

\newcommand*{\negl}{\text{negl}}
\newcommand*{\poly}{\text{poly}}
\newcommand*{\from}{\gets}
\newcommand*{\pp}{\mathrm{pp}}
\newcommand*{\acc}{\mathrm{acc}}
\newcommand*{\ToInstance}{\mathrm{\text{T\scriptsize O}\text{I\scriptsize NSTANCE}}}
\newcommand*{\Prover}{\mathrm{\text{P\scriptsize ROVER}}}
\newcommand*{\Verifier}{\mathrm{\text{V\scriptsize ERIFIER}}}
\newcommand*{\Setup}{\mathrm{\text{S{\scriptsize ETUP}}}}
\newcommand*{\Generator}{\mathrm{\text{G{\scriptsize ENERATOR}}}}
\newcommand*{\Trim}{\mathrm{\text{T{\scriptsize RIM}}}}
\newcommand*{\Commit}{\mathrm{\text{C{\scriptsize OMMIT}}}}
\newcommand*{\Decider}{\mathrm{\text{D{\scriptsize ECIDER}}}}
\newcommand*{\SNARKProver}{\mathrm{\text{SNARK}}.\Prover}
\newcommand*{\SNARKVerifier}{\mathrm{\text{SNARK}}.\Verifier}
\newcommand*{\SNARKVerifierSlow}{\mathrm{\text{SNARK}}.\mathrm{\text{V\scriptsize ERIFIER}\text{S\scriptsize LOW}}}
\newcommand*{\SNARKVerifierFast}{\mathrm{\text{SNARK}}.\mathrm{\text{V\scriptsize ERIFIER}\text{F\scriptsize AST}}}
\newcommand*{\IVCProver}{\mathrm{\text{IVC}}.\Prover}
\newcommand*{\IVCVerifier}{\mathrm{\text{IVC}}.\Verifier}
\newcommand*{\AS}{\text{AS}}
\newcommand*{\ASGenerator}{\AS.\Generator}
\newcommand*{\ASSetup}{\AS.\Setup}
\newcommand*{\ASProver}{\AS.\Prover}
\newcommand*{\ASVerifier}{\AS.\Verifier}
\newcommand*{\ASDecider}{\AS.\Decider}
\newcommand*{\PC}{\text{PC}}
\newcommand*{\PCSetup}{\PC.\Setup}
\newcommand*{\PCTrim}{\PC.\Trim}
\newcommand*{\PCCommit}{\PC.\Commit}
\newcommand*{\PCOpen}{\PC.\mathrm{\text{O\scriptsize PEN}}}
\newcommand*{\PCCheck}{\PC.\mathrm{\text{C\scriptsize HECK}}}
\newcommand*{\PCDL}{\text{PC}_{\text{DL}}}
\newcommand*{\PCDLSetup}{\PCDL.\Setup}
\newcommand*{\PCDLTrim}{\PCDL.\Trim}
\newcommand*{\PCDLCommit}{\PCDL.\Commit}
\newcommand*{\PCDLOpen}{\PCDL.\mathrm{\text{O\scriptsize PEN}}}
\newcommand*{\PCDLSuccinctCheck}{\PCDL.\mathrm{\text{S{\scriptsize UCCINCT}C{\scriptsize HECK}}}}
\newcommand*{\PCDLCheck}{\PCDL.\mathrm{\text{C\scriptsize HECK}}}
\newcommand*{\ASDL}{\text{AS}_{\text{DL}}}
\newcommand*{\ASDLSetup}{\ASDL.\Setup}
\newcommand*{\ASDLIndexer}{\ASDL.\mathrm{\text{I\scriptsize NDEXER}}}
\newcommand*{\ASDLCommonSubroutine}{\ASDL.\mathrm{\text{C{\scriptsize OMMON}S{\scriptsize UBROUTINE}}}}
\newcommand*{\ASDLProver}{\ASDL.\mathrm{\text{P\scriptsize ROVER}}}
\newcommand*{\ASDLVerifier}{\ASDL.\mathrm{\text{V\scriptsize ERIFIER}}}
\newcommand*{\ASDLDecider}{\ASDL.\mathrm{\text{D\scriptsize ECIDER}}}
\newcommand*{\CM}{\mathrm{\text{CM}}}
\newcommand*{\CMSetup}{\CM.\Setup}
\newcommand*{\CMTrim}{\CM.\Trim}
\newcommand*{\CMCommit}{\CM.\Commit}

\newcommand*\Result{\mathbf{Result}}
\newcommand*\Option{\mathbf{Option}}
\newcommand*\Acc{\mathbf{Acc}}
\newcommand*\Instance{\mathbf{Instance}}
\newcommand*\Proof{\mathbf{Proof}}
\newcommand*\Witness{\mathbf{Witness}}
\newcommand*\PublicInfo{\mathbf{PublicInputs}}
\newcommand*\PublicInputs{\mathbf{PublicInputs}}
\newcommand*\Circuit{\mathbf{Circuit}}
\newcommand*\AccHiding{\mathbf{AccHiding}}
\newcommand*\EvalProof{\mathbf{EvalProof}}


\ifLuaTeX
  \usepackage{selnolig}  % disable illegal ligatures
\fi
\usepackage{bookmark}
\IfFileExists{xurl.sty}{\usepackage{xurl}}{} % add URL line breaks if available
\urlstyle{same}
\hypersetup{
  colorlinks=true,
  linkcolor={black},
  filecolor={Maroon},
  citecolor={Blue},
  urlcolor={GbBlueDk},
  pdfcreator={LaTeX via pandoc}}

\author{}
\date{2025-01-29 - 10:45:02 UTC}

\begin{document}

\subsubsection{Lemma 3.3}\label{lemma-3.3}

Let \(F: \mathbb{N} \to \mathbb{N}\), and
\(\text{CM} = (\text{Setup}, \text{Trim}, \text{Commit})\) be a
commitment scheme. Fix a number of variables \(M \in \mathbb{N}\) and
maximum degree \(N \in \mathbb{N}\). Then for every family of (not
necessarily efficient) functions \(\{f_{pp}\}_{pp}\) and fields
\(\{\mathbb{F}_{pp}\}_{pp}\) where
\(f_{pp}: \mathcal{M}_{pp} \to \mathbb{F}^{\leq N}_{pp}[X_1, \ldots, X_M]\)
and \(|\mathbb{F}_{pp}| \geq F(\lambda)\); for every message format
\(L\) and efficient \(t\)-query oracle algorithm \(A\), the following
holds:

\[
\Pr\left[
\begin{array}{c}
p \not\equiv 0 \\
\land \, p(z) = 0
\end{array}
\middle|
\begin{array}{l}
pp \leftarrow \text{CM.Setup}(1^\lambda, L) \\
(\ell, p \in \mathcal{M}_{ck}, \omega) \leftarrow A^{\rho}(pp) \\
ck \leftarrow \text{CM.Trim}(pp, \ell) \\
C \leftarrow \text{CM.Commit}(ck, p; \omega) \\
z \in \mathbb{F}^{N}_{pp} \leftarrow \rho(C) \\
p \leftarrow f_{pp}(p)
\end{array}
\right]
\leq \sqrt{(t+1) \cdot \frac{MN}{F(\lambda)}} + \text{negl}(\lambda).
\]

If \(\text{CM}\) is perfectly binding, then the above holds also for
computationally-unbounded \(t\)-query adversaries \(A\).

Similarly to the completeness case, we consider a simplified version of
the definition of soundness in Section 4. This simpler definition
requires that the following probability is negligible for every
Polynomial-Size Adversary \(\Ac\):

\[
\Pr \left[
  \begin{array}{c|c}
    \begin{array}{c}
      \Verifier^\rho([q_i]_{i=1}^n, \acc, \pi_V) = 1, \\
      \ASDecider^\rho(\acc) = 1 \\
      \wedge \\
      \exists i \in [n],
      \, \Phi^\rho_{\PC}(\pp_\PC, i_\phi, q_i) = 0
    \end{array}
  & \quad
    \begin{aligned}
      \rho &\leftarrow \Uc(\lambda), \\
      \pp &\leftarrow \ASGenerator^\rho(1^\lambda), \\
      \pp_\PC &\leftarrow \mathcal{H}_{\PC}^\rho(\pp), \\
      (i_\phi, [q_i]_{i=1}^n, \acc, \pi_V) &\leftarrow \Ac^\rho(\pp, \pp_{\PC}) \\
      (\text{apk}, \text{avk}, \text{dk}) &\leftarrow \mathcal{I}^\rho(\text{pp}, \text{pp}_{\text{PC}}, i_\phi)
    \end{aligned}
  \end{array}
\right]
\]

Fix a polynomial-size adversary \(\Ac\) and degree bound \(D\), and
denote by \(\delta\) the above probability for these choices. We will
construct an adversary for the zero-finding game in Lemma 3.3 that wins
with probability \(\delta / 2 - \text{negl}(\lambda)\), from which it
follows that \(\delta\) is negligible (since \(q\) is superpolynomial in
\(\lambda\)).

We first describe the commitment schemes \(\CM_1, \CM_2\) used in the
zero-finding games. Both schemes have common setup and trimming
algorithms, and public parameters \(\pp\) equal to the public parameters
of \(\PCDL\) with maximum degree \(L\). The message space
\(\Mc_{\text{pp}}\) for \(CM_1\) consists of tuples \((p,
h)\), where \(p\) and \(h\) are univariate polynomials of degree at most
\(L\). Note that \(h\) is uniquely represented by
\([h_i]_{i=0}^n, \alpha\), where each \(h_i\) is a univariate polynomial
of degree \(L\), and \(\alpha
\in \Fb_q\).

The message space \(\Mc_\pp\) for \(\CM_2\) consists of lists of pairs
\([(h_i, U_i)]_{i=0}^n\), where each \(h_i\) is a univariate polynomial
of degree at most \(L\), and each \(U_i\) is a group element.

\begin{center}\rule{0.5\linewidth}{0.5pt}\end{center}

\paragraph{Algorithm:}\label{algorithm}

\begin{enumerate}
\def\labelenumi{\arabic{enumi}.}
\tightlist
\item
  \(\CM_j.\Setup^{\rho}(1^\lambda, L)\):

  \begin{itemize}
  \tightlist
  \item
    Output: \(\text{pp} \leftarrow \PCDLSetup^\rho(1^\lambda, L)\).
  \end{itemize}
\item
  \(\CM_j.\Trim^\rho(\text{pp}, n, N)\):
\end{enumerate}

\begin{itemize}
\tightlist
\item
  Compute
  \((\text{ck}_0, \text{rk}_0) \leftarrow \PCDLTrim^\rho(\text{pp}, N)\).
\item
  Output: \(\text{ck} := (\text{ck}_0, n)\).
\end{itemize}

\begin{enumerate}
\def\labelenumi{\arabic{enumi}.}
\setcounter{enumi}{2}
\tightlist
\item
  \(\CM_1.\Commit(\text{ck} = (\text{ck}_0, n), p = (p, h); r)\):

  \begin{itemize}
  \tightlist
  \item
    Commit to \(p\): \(C \leftarrow \PCDLCommit(\text{ck}_0,
    p; \omega = \perp)\). - Output: \((C, h)\).
  \end{itemize}
\item
  \(\CM_2.\Commit(\text{ck}, p = \{(h_i, U_i)\}_{i=0}^n; r)\):

  \begin{itemize}
  \tightlist
  \item
    Output: \(p\).
  \end{itemize}
\end{enumerate}

\begin{center}\rule{0.5\linewidth}{0.5pt}\end{center}

Both commitment schemes are binding. It remains to specify the families
of functions \(\{f^{(1)}_{\text{pp}}, f^{(2)}_{\text{pp}}\}\) that we
use in the respective zero-finding games.

We define:
\(f^{(1)}_{\text{pp}}(p, h = \{h_i\}_{i=0}^n) := p - \sum_{i} \alpha^i h_i,\)

and: \(f^{(2)}_{\text{pp}}(p = \{(h_i, U_i)\}_{i=0}^n):\)

\begin{enumerate}
\def\labelenumi{\arabic{enumi}.}
\tightlist
\item
  Construct the key pair
  \((\text{ck}_0, \text{rk}_0) = \PCDLTrim(\pp_\PC, \deg(h_i))\).
\item
  For each \(i \in [0, \ldots, n]\), construct a \(\PCDL\) commitment to
  \(h_i\): \(B_i \leftarrow \PCDLCommit(\text{ck}, h_i, \bot)\).
\item
  For each \(i \in [0, \ldots, n]\), compute \(a_i \in \Fb_q\) such that
  \(a_i G = U_i - B_i\).
\item
  Output the polynomial \(a(Z) = \sum_{i=0}^n a_i Z^i\).
\end{enumerate}

Both commitment schemes are binding. It remains to specify the families
of functions \(f_{\text{pp}}^{(1)}, f_{\text{pp}}^{(2)}\) that we use in
the respective zero-finding games.

We define:
\(f_{\text{pp}}^{(1)}(p, h = [h_i]_{i=0}^n) := p - \sum_{i=0}^n \alpha^i h_i,\)

and: \(f_{\text{pp}}^{(2)}(p = [(h_i, U_i)]_{i=0}^n):\)

\begin{enumerate}
\def\labelenumi{\arabic{enumi}.}
\tightlist
\item
  Construct the key pair
  \((\text{ck}_0, \text{rk}_0) \leftarrow \PCDLTrim(\text{pp}_{\text{PC}}, \deg(h_i))\).
\item
  For each \(i \in \{0, \ldots, n\}\), construct a \(\PCDL\) commitment
  to \(h_i\): \(B_i \leftarrow \PCDLCommit(\text{ck}_0, h_i, \bot)\).
\item
  For each \(i \in \{0, \ldots, n\}\), compute \(a_i \in \mathbb{F}_q\)
  such that \(a_i G = U_i - B_i\).
\item
  Output the polynomial \(a(Z) := \sum_{i=0}^n a_i Z^i\).
\end{enumerate}

\begin{center}\rule{0.5\linewidth}{0.5pt}\end{center}

We next describe an adversary \(C\) against \(\PCDL\), which simply runs
the soundness experiment for the accumulation scheme and outputs
\(\text{acc}\) as output by \(\mathcal{A}\). For convenience, we also
have \(C\) output \([q_i]_{i=1}^n\) and \(\pi_V\); this will be ignored
by the extractor. \textbf{\(C^\rho(\text{pp}_{\text{PC}})\):}

\begin{enumerate}
\def\labelenumi{\arabic{enumi}.}
\tightlist
\item
  Set AS public parameters \(\text{pp}_{\text{AS}} := 1^\lambda\).
\item
  Compute \((i_\phi, [q_i]_{i=1}^n, \text{acc}, \pi_V) \leftarrow
  \mathcal{A}^\rho(\text{pp}_{\text{AS}}, \text{pp}_{\text{PC}})\).
\item
  Parse \(i_\phi\) as the degree bound \(N\).
\item
  Output
  \((N, \text{acc} = ((C, d, z, v), \pi); [q_i]_{i=1}^n, \pi_V)\).
\end{enumerate}

\begin{center}\rule{0.5\linewidth}{0.5pt}\end{center}

We use the extractor \(\mathcal{E}_C\) corresponding to \(C\) to
construct adversaries \(B_1, B_2\) for zero-finding games against
\((CM_1, \{f_{\text{pp}}^{(1)}\}_{\text{pp}})\), \((CM_2,
\{f_{\text{pp}}^{(2)}\}_{\text{pp}})\) respectively, with \(L = D\)
where \(D =
\text{poly}(\lambda)\) is the maximum degree parameter as in the
soundness experiment for the accumulation scheme.

\textbf{\(B_j^\rho(\text{pp})\):}

\begin{enumerate}
\def\labelenumi{\arabic{enumi}.}
\tightlist
\item
  Compute \((N, \text{acc}, [q_i]_{i=1}^n, \pi_V) \leftarrow
  C^\rho(\text{pp})\).
\item
  Parse \([a_i]_{i=1}^n\) as \(((C_i, d_i, z_i, v_i), \pi_i)\) and
  \(\pi_V\) as \((h_0, U_0, w)\).
\item
  Compute \(p \leftarrow \mathcal{E}_C^\rho(\text{pp})\).
\item
  For each \(i \in [n]\), obtain \(h_i\) and \(U_i\) from \(\pi_i\).
\item
  Compute \(\alpha := \rho_1([(h_i, U_i)]_{i=0}^n)\).
\item
  If \(j = 1\), output \(((n, N), (p, h := ([h_i]_{i=0}^n)))\). If
  \(j = 2\), output \(((n, N), ([(h_i, U_i)]_{i=0}^n))\). ---
\end{enumerate}

We show that either \(B_1\) or \(B_2\) wins its respective zero-finding
game with probability at least \(\delta / 2 - \text{negl}(\lambda)\).

Since \(D\) accepts with probability \(\delta\), and by the extraction
property of \(\PCDL\), the following holds with probability at least
\(\delta -
\text{negl}(\lambda)\): \(\mathcal{E}_C\) outputs a polynomial \(p\)
such that \(C\) is a commitment to \(p\) with randomness \(w\) (and so
\(C\) is a deterministic commitment to \(p\)), \(p(z) = v\), and
\(\deg(p) \leq d\); and, moreover,
\((\text{acc}, \{q_i\}_{i=1}^n, \pi_V)\) satisfies the left-hand side of
Eq. (7). This latter point implies that, parsing \(q_i\) as
\((C_i, d_i, z_i,
v_i, \pi_i)\) and letting \((h_i, U_i) := \PCDLSuccinctCheck^\rho
(\text{rk}, C_i, d_i, z_i, v_i, \pi_i)\):

\begin{itemize}
\tightlist
\item
  Since
  \(\mathcal{V}^\rho(\text{avk}, [q_i]_{i=1}^n, \text{acc}, \pi_V)\)
  accepts, the following are true:

  \begin{enumerate}
  \def\labelenumi{\arabic{enumi}.}
  \tightlist
  \item
    For each \(i \in [n]\), \(\PCDLSuccinctCheck\) accepts.
  \item
    \(U_0\) is a commitment to \(h_0\).
  \item
    Parsing \(\text{acc}\) as \((C, d, z, v), \pi\) and setting
    \(\alpha :=
    \rho_1([(h_i, U_i)]_{i=0}^n)\), we have that
    \(z = \rho_1(C, [h_i]_{i=0}^n)\), \(C = \sum_{i=0}^n \alpha^i U_i\),
    and \(v = \sum_{i=0}^n \alpha^i h_i(z)\).
  \end{enumerate}
\end{itemize}

For some \(i \in [n]\), \(\Phi_{PC}^\rho(\text{pp}_{PC}, i_\phi, q_i)
= \PCDLCheck^\rho(\text{rk}, (C_i, d_i, z_i, v_i), \pi_i)
= 0\). By construction (see Appendix A.2), this implies that either
\(\PCDLSuccinctCheck\) rejects, or the group element \(U_i\) is not a
commitment to \(h_i\).

\begin{center}\rule{0.5\linewidth}{0.5pt}\end{center}

The above tells us that there exists some \(i \in [n]\) such that
\(U_i\) is not a commitment to \(h_i\). In other words, if we define
\(B_i :=
\PCDLCommit(\text{ck}, h_i)\), then there exists an \(i \in [n]\) such
that \(U_i \neq B_i\). Letting \(a_i \in \mathbb{F}_q\) be such that
\(a_i
G = U_i - B_i\), we deduce that the polynomial
\(a(Z) = \sum_{i=0}^n a_i Z^i\) is not identically zero.

There are then two cases:

\begin{enumerate}
\def\labelenumi{\arabic{enumi}.}
\item
  \(C \neq \sum_{i=0}^n \alpha^i B_i\). Then since \(C\) is a commitment
  to \(p\), \(p(X) - h(X)\) is not identically zero, but
  \(p(z) = h(z)\). Hence \(B_1\) wins the zero-finding game against
  \((CM_1, \{f_{pp}^{(1)}\}_{pp})\).
\item
  \(C = \sum_{i=0}^n \alpha^i B_i\). Then since
  \(C = \sum_{i=0}^n \alpha^i U_i\), \(a(Z)\) is a zero of the
  polynomial \(a(Z)\). Hence \(B_2\) wins the zero-finding game against
  \((CM_2, \{f_{pp}^{(2)}\}_{pp})\).
\end{enumerate}

Since at least one of these two cases occurs with probability at least
\(\delta / 2 - \text{negl}(\lambda)\), the claim follows.

\end{document}
